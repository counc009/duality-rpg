\documentclass[titlepage]{article}

\usepackage[utf8]{inputenc}
\usepackage[top=2cm,right=2cm,left=2cm,bottom=2cm]{geometry}
\usepackage{amsfonts}
\usepackage{amsmath}
\usepackage{amssymb}
\usepackage{amsthm}
\usepackage[english]{babel}
\usepackage{fancyhdr}
\usepackage[dvipsnames]{xcolor}
\usepackage{hyperref}
\usepackage{comment}

\usepackage{yfonts}
\usepackage{pdfpages}

\pagestyle{fancy}
\fancyhf{}
\rhead{For Version 0.1.1 (January 4, 2025)}
\lhead{In the Celeste Forest}

\renewcommand{\footrulewidth}{1pt}
\rfoot{\thepage}

\begin{document}

\yinipar{O}ur story takes place in the world of \textbf{Faero}.
To the far south of the world, we find the \textbf{Dark Forest}, a forest once controlled by faerie spirits but now overrun by corrupted creatures of the wizard \textbf{Eqihr}.
Creatures of all sorts roam the Dark Forest: monsters with 2 heads and 7 eyes which can take any shape they desire, felines the size of elephants, frogs big enough to swallow a child a whole, just to name a few.
The ``corruption'' these creature seem to emanate turns the land black, killing plants and animals in its wake and leaving corrupted beings, husks of what once lived there, in its wake.

But this story begins far to the north of the Dark Forest, in a different forest.
We find ourselves in the \textbf{Celeste Forest}.
You have been hired to lead a large caravan of about 300 people.
You've been traveling north for about a week now, fleeing from the town of \textbf{Draycott} which the Dark Forest had been rapidly encroaching on and by now now may have been fully consumed.

Your trip began along paved roads with large carts but the path you're following has led you deep into the Celeste Forest.
Most of the carts have been left behind, with the exception of a few hand carts.
A few people ride on horses or donkeys, but most walk many carrying all their remaining possessions.

Your destination is the city of \textbf{Citimolaure}, a city told of in the great legends of this world.
If the stories are to be believed, in times long past the town survived the sieges of wizards who sought to destroy it so they could breach the city walls to steal the secrets of alchemy which had enabled the town to transform its walls from stone into pure gold.
In the ancient elvish, the name Citimolaure translates as ``City of Gold.''

\section*{The Caravan}
The caravan's de-facto leader is \emph{Urifina Feyldo}, an old elven woman.
For an elf to appear old, she must be truly ancient.
She rides on a horse and moves throughout the caravan, checking in on people, checking that have supplies and trying to keep spirits up by telling stories of great adventures she claims to have undertaken.

\section{The Cliff}
This morning, you find yourselves traveling about half a mile ahead of the caravan, scouting ahead to find the best path.
The forest here is thick, you can't see more than about 20 feet ahead of you between all the trees and undergrowth.
As it's approaching midday, the forest seems to slowly be starting to thin and up ahead you catch sight of a tall nearly vertical cliff that stands in your way.
As you get closer, you can see that the cliff here seems about 200 feet tall and seems to extend both ways as far as you can see.

\begin{itemize}
  \item[-] This cliff is told of in legends as the \textbf{Great Barrier}, a tall cliff edge that runs down the middle of this world.
  \item[-] The height of the cliff varies, there are parts where it gets as low as maybe 30 feet but others that are over 1000 feet tall.
    Searching the nearby area (difficulty Medium) you can find a shorter section (about 100 on Mixed/Good and 50 on Very Good).
  \item[-] Someone climbing up the cliff is a Hard roll (this may be reduced if they find a better spot).
    Trying to throw something up to the top of the cliff is a Very Hard roll.
  \item[-] Getting everyone up the cliff, however it's done, will be a long task, it should be a Collective Roll with a \emph{Magnitude} of 5.
    The default Difficulty is Medium, if the party finds a shorter section of cliff this should be lowered.
\end{itemize}

\section{The Corruption}
Continuing onwards, later into the day, the sound of birds, insects, and critters from the forest begin to change.
Some sounds are quieting, others though are louder but sound just a little bit off.

The party should roll to Perceive (or similar skills) and the level of success determines how quickly they notice the presence of the Corruption.
On Bad or Very Bad results, the party is immediately in danger from attack while on a Mixed or better result they'll have some notice.
On a Mixed result, they notice something prowling nearby, while on Good or Better they notice signs of corruption before anything discovers them.

\begin{itemize}
  \item[-] There is a large region of corruption in this forest.
    The party will recognize it from what they have heard of the corruption, if not from personal experience.
  \item[-] This region of corruption is about three miles across (in the north-south direction) and about 50 miles wide (in the east-west direction).
  \item[-] In the center is a lake, about 2 miles across and 40 miles wide.
    It is filled with a black ichor, touching the liquid produces a burning sensation and seems to blacken whatever is immersed.
    If left in contact for too long, the ichor leaches into anything living, corrupting it.
\end{itemize}

\paragraph{Corruption}
When a creature is corrupted it becomes able to corrupt other creatures (plants, animals, and people) and gains an Advantage anytime it does so.
It feels a connection to the corruption around it, and the ability to sense things through it; as it becomes more corrupted by spending more time in a corrupted environment this connection becomes stronger to the point where it is just part of the environment now.
A slightly corrupted being will feel a pull back as they try to leave a corrupted area as it will break the connection; this may require a Willpower roll at a Difficulty determined by the extent of the corruption.

\begin{itemize}
  \item[-] Surrounding the lake, the forest is corrupted with the trees nearest to the shore being fully black, almost as if made from the ichor in the lake and are slightly animated and act as guards of the lake (they must be pushed or cut through).
  \item[-] Further away from the lake, only some trees have turned black while others are dead, and further from the center there are more living and green trees and fewer black and dead ones.
  \item[-] Other vegetation is similarly affected, though it stretches further than the corruption of the trees.
\end{itemize}

\paragraph{Corrupted Mushroom}
Extending farthest out from the lake are small pure black mushrooms that also seem to be made of jello and carry a tiny bit of life.
Because of how they jiggle, it is quite difficult to inflict damage on them.
\\
\textbf{Life}: 5, \textbf{Block}: 3, \textbf{Size}: Tiny, \textbf{Speed}: None
\\
\textbf{Defend}: -1, \textbf{Weapon, Spores [Ranged, Piercing]}: +2, \textbf{Weapon, Corrupting Poison [Melee]}: +1
\\
When a Corrupted Mushroom succeeds on a Corrupting Poison attack and inflicts damage, the target becomes slightly corrupted and the GM gains an Advantage.

\paragraph{Corrupted Rat}
Pitch black, corrupted rats scurry quickly through the forest.
\\
\textbf{Life}: 10, \textbf{Block}: 0, \textbf{Size}: Small, \textbf{Speed}: Fast
\\
\textbf{Defend}: +1, \textbf{Weapon, Corrupting Bite [Melee]}: +2

\paragraph{Corrupted Panther}
The corrupting panther prowls the forest and come after anything that comes within its domain.
\\
\textbf{Life}: 20, \textbf{Block}: 1, \textbf{Size}: Large, \textbf{Speed}: Fast
\\
\textbf{Defend}: +2, \textbf{Weapon, Claws [Melee, 2d8]}: +3, \textbf{Weapon, Corrupting Bite [Melee]}: +3

\paragraph{Corrupted Guardian Tree}
The guarding trees reach out and block any path down to the edges of the lake.
Creatures must hack or pass their way through, giving the trees an opportunity to scratch and corrupt.
The trees are entirely black and almost seem like they might slowly drip almost tar-like.
Their branches are rough and jagged, and almost seem jointed.
\\
\textbf{Life}: 30, \textbf{Block}: 3, \textbf{Size}: Large, \textbf{Speed}: None
\\
\textbf{Defend}: -1, \textbf{Weapon, Branches [Melee, Grapling]}: +2, \textbf{Weapon, Corrupting Vines [Melee]}: +3
\\
A Corrupted Guardian Tree can grab and pull creatures into its branches.
When it does so, the vines seem to suddenly appear and can attack and corrupt grappled creatures.

\begin{itemize}
  \item[-] It is difficult to spot (Hard roll) but in the middle of the lake is a mirrored tower from which a small flame escapes every few minutes.
\end{itemize}

\section{The Strangers}
When the party retreats back out of the corrupted area, nearly as soon as they are out someone spots a large tree that has a small door and window carved into the base.
The window is covered by thin cloth but light spills out.
They simultaneously hear distant arrhythmic thuds.

\subsection{The Witch (Alina Everbleed)}
Alina is quite sweet to people she knows and likes but otherwise plays the part of the witch/hag.
She is dressed all in black with a tall pointed hat, her eyes are lime green and seem to glow faintly, and her skin is nearly snow white.
Somehow you can tell that she is ancient yet she looks no older than 30.

As soon as someone approaches the hut, the door swings open on its own.
Inside is a warmly lit interior; the walls appear to be the inside of the tree but are covered in pine bows, dried ivy, various branches with raspberries, blackberries, and blueberries, and other dried herbs and flowers.
Just inside the door is an overstuffed couch turned towards a crackling fire with a tea kettle set above it.
Nobody is visible.

As you enter, the room seems to stretch impossibly and you see the witch seated in a creaky wooden chair facing the back of the couch.
With the twist of a long finger, the couch turns to face her, ``Come sit...''
In the seat she looks small, not more than 5 feet tall, but if she ever stands she is over 7 feet tall.

``Welcome to my humble abode, whatever it is that you seek I can grant to you, for the right price of course.
Gold and riches, fame and glory, even love I can provide to you if you but tell me what it is you seek and I'll offer you a fair price for whatever it is.''

If the party asks for help from Alina in getting across the corruption, she will tell them she can certainly do that, for the right price and then tell them (depending on whether they've brought Mekkosh with them):
\begin{itemize}
  \item[-] ``Bring me a heart, an angel, and a storm''
    \begin{itemize}
      \item[$\circ$] ``angle'' = statue = rock
      \item[$\circ$] ``heart'' = heartwood = wood
      \item[$\circ$] ``storm'' = clouds = water
    \end{itemize}
  \item[-] ``Bring me the roots of a mountain, the mouth of a tree, and the peak of a river; or something like that.''
    \begin{itemize}
      \item[$\circ$] ``peak of a mountain'' = rock
      \item[$\circ$] ``roots of a tree'' = wood
      \item[$\circ$] ``mouth of a river'' = water
    \end{itemize}
\end{itemize}

\subsection{The Orc (Mekkosh)}
Mekkosh lives about a mile from Alina, he has a small hut with a large garden surrounding it.
The hut is made of wood but covered by a thick blanket of ivy, moss, and flowers.
Inside the hut are two rooms, one with a bed and a small night stand with a drawer that seems to be overfilled with seeds.
The front room has a fireplace, a large black cauldron, a few wooden chairs, and a table.
The garden grows both food (lots of carrots, beets, turnips, beans, and tomatoes) and flowers (lilies, tulips, daffodils, and many others).

Mekkosh is about 9 feet tall with dark green skin and large tusks.
He's a druid and spends most of his time tending to the garden.
He's out in the garden splitting wood with an ax and is excited to see visitors and is curious about what brings them to the area.

Alina and Mekkosh are good friends and Mekkosh can bring the party to her and will at times interrupt to tone her down or get her to give the more clear version of her request.

\section{The Traversal}
Once the party pays Alina (and they don't have to return with the items she requested, as long as they can explain why what they brought satisfies her requests she will accept it), she stands up to her full height and begins walking far quicker than it seems she should possibly be able to move.

She directs the group to gather themselves and the rest of the caravan at the edge of the lake.
She cracks her knuckles and with a flick of her wrist the surface suddenly freezes into a bridge of ice.
She shakes her hand for a moment, ``not quite...'' and then it suddenly transforms into a wooden bridge, ``Quickly now.''

As they approach about half way across the lake, the party will notice the Tower if they haven't already.
To try to move across the bridge without detection, is a Medium cooperative roll or Hard if the party had not noticed the tower before.
A Fire Hawk will depart the tower when the group is about three-quarters of the way across, and if they had been noticed the Hawk will come to attack.

\paragraph{Fire Hawk}
A massive hawk shaped bird, about the size of a person, it stands about 5 feet tall perched (ignoring its tail).
Its feathers are nearly all black, almost like a crow, except for its underbelly which seems to crackle with black and orange flames.
\\
\textbf{Life}: 15, \textbf{Block}: 0, \textbf{Size}: Medium, \textbf{Speed}: Fast (fly)
\\
\textbf{Defend}: +2, \textbf{Weapon, Claws [Melee, Reach]}: +2, \textbf{Weapon, fire breath [Ranged, Multihit]}: +2
\\
The fire hawk, if it noticed the caravan, will attempt to go after the masses of people first with fire and then grab those who survive and try to drop them into the lake.

\section{Conclusion}
You make it across the lake and continue traveling north.
Slowly, the corruption of the forest recedes.
You find yourselves in a normal forest again, though this one is filled with pines, though the forest floor seems almost swept clean of their needles.
The sight lines in this forest are much improved, as if someone tended to it, and you can see a handful of deer wandering in the distance as you travel.

The journey continues, day after day; it's not easy going necessarily but the hardest parts of the journey are behind you.
It's about a week later, approaching sunset and almost time to setup camp for the night when you catch sight of a glimpse of what looks like gold on the horizon.

Rushing forward, you can start to make out the details of a skyline: a city. A city of gold.
And that is where we will end this adventure.

\end{document}
