\section{Items}\label{sec:items}\index{Item}\index{Weapon|see{Item}}\index{Armor|see{Item}}\index{Shield|see{Item}}
Beyond a character's own abilities, they are also generally equipped with certain equipment that aids in their adventures.
Much of the equipment is not noteworthy and can generally be assumed, but the key pieces of equipment are \textbf{Items} and given their own statistics.
These statistics define bonuses and other features that the \emph{Item} may grant to the character using it.
\emph{Items} fall into one of two categories: \textbf{Weapons} and \textbf{Relics}.
Weapons, including defensive items like shields and armor, are associated with a particular Combat Style the character has and grant bonuses to combat abilities.
Relics are any non-Weapon items and they grant the user an additional \Experience{} that reflects the nature or possibly history of the Relic.
Both Weapons and Relics can also grant other benefits, including bonuses to other \emph{Attributes} or \emph{Specializations}, additional \emph{Experiences}, or even new \emph{Abilities}.

\subsection{Creating an Item}
At character creation a character receives one Item for free, but otherwise creating an Item costs 1 XP which reflects the costs of time, money, and resources that procuring or creating the Item takes.
For this cost, the item that is created is a \emph{Basic Item}, the properties of which depend on whether it is a \emph{Weapon} or \emph{Relic}.
When the item is created you can also spend XP to add additional features or improve on the Items basic features.

\subsubsection{Creating a Basic Weapon}\index{Item!Weapon}
To create a basic weapon, choose one of the character's Combat Styles to associate it with.
If the chosen Combat Style is Offensive, the item grants a $+1$ bonus to attacks and 1 additional damage die.
If the chosen Combat Style is Defensive, the item grants a $+1$ bonus to defend and $+1$ bonus to \Block{}.

\subsubsection{Creating a Basic Relic}\index{Item!Relic}
To create a basic relic, define the \Experience{} that the Relic grants a $+1$ bonus for.
At the beginning of each session, the Relic gains an \Advantage{} (if it does not have one already) that can only be spent on activating its \Experience{}.

\subsubsection{Finishing an Item}
Finally, additional features can be purchased for an Item using XP.
\begin{itemize}
  \item To increase a bonus (to an \Attribute{} or \Specialization, to \Block{}, to rolls to attack or defend, or to an \Experience) costs a number of XP equal to the \emph{total bonus} the item will offer after the increase.
    This means that if an item already grants two $+1$ bonuses increasing one of them to $+2$ costs 3 XP since the total bonus will now be $+3$.
    The cost to add a new bonus of any of these kinds (including through adding a new \Experience) is the same.
    The maximum bonus to an \Attribute{}, \Specialization, \Block, or \Experience{} is $+2$ and the maximum bonus to attack or defend is the same as the maximum bonus for a character's \emph{Specializations}.
  \item To add a bonus to \Life{} costs 2 XP per point up to a maximum bonus of $+ 10$
  \item To add a bonus to \Recovery{} costs 4 XP per point up to a maximum bonus of $+ 5$
  \item To add an \Ability{} costs the cost of that \Ability{}
\end{itemize}
For weapons, the following options can also be added
\begin{itemize}
  \item To add an Combat Style that the weapon can be used with costs 3 XP.
  \item An additional damage die can be added for $3 \times$ the new number of damage dice granted by the item. A weapon can grant a maximum of 3 damage dice.
\end{itemize}

\subsection{Other Item Rules}\index{Item}
There are a few other Item rules, especially for governing how they interact.
First, if a character has two Items which grant a bonus to the same thing (the same \Attribute{} or \Specialization) these bonuses do not stack, instead only the higher bonus applies.
Similarly, if a character has multiple different weapons, the attack bonus only applies when using that specific weapon.
For defensive weapons, similarly only the highest bonus to defend applies.

Additionally, if a character has more than 2 items in use (a character can hold onto other item but if they are not in use they would require several hours to be able to use them) they must hold onto 1 XP for each item past 2 they have.

Finally, sometimes a character outgrows a particular item, either they find something better or they have changed enough that the item no longer makes sense for them.
In such case they can sell the item to regain all XP that they had invested in creating the item (make sure to track this for all Items); because this represents, in game, finding a buyer or perhaps melting down a weapon to forge it into something else.
The GM may not allow an Item to be sold in a circumstance where it does not fit the narrative.
