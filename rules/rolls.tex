\section{Rolls}
When a Player Character (PC) attempts to do something that is not certain, they must roll for it.
There are two types of roll: opposed rolls for when another character is resisting the effort and unopposed rolls for when success is not guaranteed but nobody is actively resisting the effort.
The result of a roll is measured by an \textbf{Outcome}\index{Outcome}: \VeryBad, \Bad, \Mixed, \Good, or \VeryGood.

When the \Outcome{} of a roll is \Mixed{} or better (\Mixed, \Good, or \VeryGood), the PC achieves what they rolled for; on a \VeryGood{} outcome they either achieve something extra or gain an \Advantage.
When the \Outcome{} is \Mixed{} or worse (\Mixed, \Bad, or \VeryBad), their is either a negative repercussion, and a particularly bad repercussion on a \VeryBad{} result, or the GM gains an \Advantage, or two on a \VeryBad{} result.

The GM may declare any roll to be \textbf{Dangerous}\index{Dangerous} in which case a \Bad{} outcome is treated as a \VeryBad{} outcome.

\subsection{Unopposed Rolls}
For an unopposed roll, the GM will select the appropriate \Skill{} for the task and set a \textbf{Difficulty}\index{Difficulty} for the task, which is \emph{Very Easy}, \emph{Easy}, \emph{Medium}, \emph{Hard}, or \emph{Very Hard}.
The player then rolls 3d20 and generally takes the middle value and then adds their \Skill{} bonus and any relevant bonuses to the roll; this total is then compared to Table~\ref{tab:unopposed-outcome} to determine the \Outcome.

\begin{table}
  \centering
  \begin{tabular}{|c|c|c|c|c|c|} \hline
   \backslashbox{\bf Difficulty}{\bf Outcome}
        &\textbf{Very Bad} &\textbf{Bad} &\textbf{Mixed} &\textbf{Good} &\textbf{Very Good} \\ \hline
    \textbf{Very Easy} &-- &$\leq 0$ &$1 - 4$ &$5 - 8$ &$\geq 9$ \\ \hline
    \textbf{Easy} &$\leq 0$ &$1 - 4$ &$5 - 8$ &$9 - 12$ &$\geq 13$ \\ \hline
    \textbf{Medium} &$\leq 4$ &$5 - 8$ &$9 - 12$ &$13 - 16$ &$\geq 17$ \\ \hline
    \textbf{Hard} &$\leq 8$ &$9 - 12$ &$13 - 16$ &$17 - 20$ &$\geq 21$ \\ \hline
    \textbf{Very Hard} &$\leq 12$ &$13 - 16$ &$17 - 20$ &$21 - 24$ &$\geq 25$ \\ \hline
  \end{tabular}
  \caption{Outcome Table for Unopposed Rolls}
  \label{tab:unopposed-outcome}
\end{table}

\subsection{Opposed Rolls}
For an opposed roll, both the Player and GM roll and add their appropriate \Skill{} bonuses and any other relevant bonuses as well.
Then, the Player's result is computed as their roll minus the GM's roll and this is compared to Table~\ref{tab:opposed-outcome}.
Similarly, the GM's result is computed as their roll minus the Player's and this is compared to Table~\ref{tab:opposed-outcome}.

\begin{table}
  \centering
  \begin{tabular}{|c|c|c|c|c|c|} \hline
    \textbf{Difference} &$\leq \text{-}6$ &$\text{-}5 - \text{-}2$ &$\text{-}1 - 1$ &$2 - 5$ &$\geq 6$ \\ \hline
    \textbf{Outcome} &Very Bad &Bad &Mixed &Good &Very Good \\ \hline
  \end{tabular}
  \caption{Outcome Table for Opposed Rolls}
  \label{tab:opposed-outcome}
\end{table}

\subsection{Group Rolls}
Groups rolls, for circumstances where multiple characters are working together on something or all doing something where failures can impact each other, have slightly different rules.
There are two types of Group rolls, whose rules are detailed below.

\subsubsection{Collective Rolls}
A Collective Roll is used when characters are working together on a large task that could not be reasonably be completed by just one.
For example, a Collective Roll might be used when a group of PCs is researching in a large library looking for information pertaining to the location of an ancient ruin.
In a Collective Roll, the GM may also allow multiple rounds of rolls to be collected together for tasks that might take more time and effort than just the single roll.

For a Collective Roll, the GM sets a \Difficulty{} and a \textbf{Magnitude}\index{Magnitude} (a whole number).
The characters then each roll as normal, but instead of converting their roll to an outcome, they instead total their rolls together.
This total is then divided by the \emph{Magnitude} set by the GM (always rounding towards zero) and then this result is compared to Table~\ref{tab:unopposed-outcome} to determine the group's outcome of the roll.

In our example of searching a library, the GM may determine that this task is relatively easy and risk free and set a \Difficulty{} of \emph{Easy} but because of the massive size of the library determine that the \emph{Magnitude} is 10.
Then, if the party rolls a total of 48, we divide this by the \emph{Magnitude} to get a 4, which is a \Bad{} outcome.

When a GM allows multiple rounds of rolls, the party totals their rolls from each round together and then divides by the \emph{Magnitude} to determine the \Outcome{} for each round.
While a \Bad{} or \VeryBad{} result on a round of rolls is not a failure of the entire effort, the GM may still impose negative outcomes or gain \Advantage{} for each round until the group reaches a \Mixed{} or better result.

If the GM for the party searching the library allows the party another round, the GM may collect an \Advantage{} for the failure on the first round (or impose some negative outcome).
The party then rolls a total of 33 on the next round, for a total of 81 now, which is an 8 after division by the \emph{Magnitude} and is therefore a \Mixed{} result now.
The party therefore has now found the information they are looking for, but the GM may take another \Advantage{} or impose a negative outcome.

\subsubsection{Cooperative Rolls}
A Cooperative Roll is used when characters are each performing similar actions with similar goals but failures by some characters may impact the entire group.
For example, a Cooperative Roll might be used when a group of PCs is sneaking into the collections of a museum seeking an artifact not on display.

For a Cooperative Roll, the GM sets a \Difficulty{} for each character which each character rolls against.
This \Difficulty{} and the \Skill{} used for the roll is often the same for all characters involved but not necessarily.
Then, the total successes and failures are totaled, where a \VeryBad{} counts as $-2$, \Bad{} as $-1$, \Mixed{} as $0$, \Good{} as $+1$, and \VeryGood{} as $+2$.
This total then is converted into an \Outcome{} in the same manner, so a result of $-2$ or less is \VeryBad, $-1$ is \Bad, $0$ is \Mixed, $1$ is \Good, and $2$ or higher is \VeryGood.

For our party attempting to sneak into the museum collections, the GM may set a \Difficulty{} of \emph{Hard} and have each PC roll for stealth.
If the party rolls two \Bad, one \Mixed, and one \VeryGood{} result, the outcome of the roll is \Mixed.
If, instead, they had rolled two \Bad, one \Good, and one \VeryGood{} the outcome would be \Good.

\subsection{Helping Out}
A character can \textbf{Help}\index{Help} another creature on a roll with that creature's and GM's approval.
Based on how they are helping out, the GM will pick a skill for the helper to roll; this will generally be against a \emph{Medium} difficulty though the GM may adjust this for particularly difficult tasks or when the assistance is difficult to render.
On a \Mixed{} or better result the creature being helped gains an \Advantage.

\subsection{Increased and Decreased Rolls}
In some circumstances a character's roll may be \textbf{Increased}\index{Increase} or \textbf{Decreased}\index{Decrease}.
\emph{Increases} and \emph{Decreases} can stack and cancel each other out, so one \Increase{} and one \Decrease{} cancel out, while two \emph{Increases} and one \Decrease{} cancel out to one \Increase.

On a roll with one \Increase{} the highest of the 3d20 is read, while on a roll with one \Decrease{} the lowest of the 3d20 is read.
On a roll with two or more (net) \emph{Increases} it is treated as if a 20 had been rolled and on a roll with two or more \emph{Decreases} as if a 1 had been rolled.
