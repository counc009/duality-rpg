\section{Character Statistics}
Characters are defined by their \emph{Attributes}, \emph{Specializations}, \emph{Experiences}, \emph{Abilities}, and \emph{Combat Styles}.
\emph{Attributes} define the character's broad statistics as well as their movement, wealth, and ability to take damage.
\emph{Specializations} define specific skills that the character possesses and the nature of those skills.
\emph{Experiences} reflect the character's past and offer bonuses to rolls in certain circumstances.
\emph{Abilities} provide unique mechanics for a character.
\emph{Combat Styles} define a character's fighting style, both offensive and defensive.

\subsection{Attributes}
Character \textbf{Attributes}\index{Attribute} represent the character's basic abilities.
Each character has four kinds of attributes that describe: their base statistics, their health, their movement abilities, and their wealth.

\subsubsection{Basic Statistics}
A character's basic statistics are defined by six \emph{Attributes}: \textbf{Strength}\index{Strength|see{Attribute}}, \textbf{Finesse}\index{Finesse|see{Attribute}}, \textbf{Willpower}\index{Willpower|see{Attribute}}, \textbf{Instinct}\index{Instinct|see{Attribute}}, \textbf{Presence}\index{Presence|see{Attribute}}, and \textbf{Knowledge}\index{Knowledge|see{Attribute}}.
These \emph{Attributes} each have a numeric value between $-1$ and $+2$ (for starting characters) and may increase to $+3$ when a character levels up.

These \Attribute{} values provide bonuses that are applied to rolls.
The GM should pick the \Attribute{} that makes most sense for the roll:
\begin{itemize}
  \item \textbf{Strength}\index{Attribute!Strength}: A character's physical strength; their ability to lift, carry, climb.
  \item \textbf{Finesse}\index{Attribute!Finesse}: A character's manual dexterity and control and their ability to move carefully and quietly.
  \item \textbf{Willpower}\index{Attribute!Willpower}: A character's will to live; their ability to push off death and rely on pure luck.
  \item \textbf{Instinct}\index{Attribute!Instinct}: A character's senses and intuition; their ability to perceive, sense motive, evaluate.
  \item \textbf{Presence}\index{Attribute!Presence}: A character's person skills; their ability to convince or deceive, to put on a performance.
  \item \textbf{Knowledge}\index{Attribute!Knowledge}: A character's knowledge and experience; their ability to recall and analyze.
\end{itemize}

\subsubsection{Health}
There are three health related \emph{Attributes}: \textbf{Life}\index{Life|see{Attribute}}\index{Attribute!Life}, \textbf{Recovery}\index{Recovery|see{Attribute}}\index{Attribute!Recovery}, and \textbf{Block}\index{Block|see{Attribute}}\index{Attribute!Block}.
A creature's \Life{} defines how much damage it can it until it falls unconscious or dies.
Its \Recovery{} defines how much \Life{} a creature regains when it heals, rests, and recuperates.
\Block{} defines how much damage a creature can just absorb without any impact, as described later \Block{} is subtracted from damage dealt to a creature before subtracting the damage from its \Life.

A character's \emph{Maximum Life} has a numeric score between 1 and 40, its \Recovery{} a numeric score between 1 and 20, and its \Block{} a numeric score between 0 and 2.
Other creatures may have values outside of these ranges.

\subsubsection{Movement}
Each creature has a \textbf{Speed}\index{Speed|see{Attribute}}\index{Attribute!Speed} which determines how fast it moves and has \emph{Attributes} which determine by what means.
A creature's \emph{Speed} is one of \emph{Slow}, \emph{Normal}, and \emph{Fast} which abstractly define how quickly it can move.
Note that these speeds should be interpreted in the full context of a creature, a \emph{Slow} dragon may still fly faster than a \emph{Fast} snail or even mouse can move, though alternatively a \emph{Fast} dragon may still walk slower than a \emph{Normal} mouse can.

By default a creature's speed is its \emph{Walking} speed, but a creature may also be able \emph{Climb}, \emph{Burrow}, \emph{Swim}, and \emph{Fly}; each of these is its own \Attribute{} a character can posses which indicates whether it can move in that manner.

\subsubsection{Wealth}
The \textbf{Wealth}\index{Wealth|see{Attribute}}\index{Attribute!Wealth} \emph{Attribute} describes how much money a creature has and can easily spend.
Creatures with a high \emph{Wealth} are able to make larger purchases, without expending other meta-resources, than a creature with low \emph{Wealth}.
\emph{Wealth} has one of the values in Table~\ref{tab:wealth} which also describes the sorts of things that can be bought with that level.

\begin{table}
  \centering
  \begin{tabular}{|c|l|} \hline
    \textbf{Wealth Value} &\textbf{Example Purchases} \\ \hline
    None &nothing \\ \hline
    Wretched &basic food and lodging daily \\ \hline
    Squalid &basic (though sturdy) supplies like rope, arrows, and tents \\ \hline
    Modest &rare supplies like climbing equipment, a day of unskilled labor, a beast of burden \\ \hline
    Comfortable &a sailboat, a comfortable wagon, a day of skilled labor \\ \hline
    Luxurious &a mercenary, an abandoned warehouse or small property \\ \hline
  \end{tabular}
  \caption{Wealth\index{Attribute!Wealth}}
  \label{tab:wealth}
\end{table}

\subsection{Specializations}
\textbf{Specializations}\index{Specialization} represent talents and abilities a creature has.
There are a total of sixty possible specializations, each described by the combination of a verb and noun and each character may possess a bonus associated with the specializations which can be used on relevant rolls and a \emph{tag} describing special mechanics for the character when they use that specialization.
The verbs and nouns that define \emph{Specializations} are:
\begin{itemize}
  \item \textbf{Verbs}: \emph{Control}, \emph{Create}, \emph{Destroy}, \emph{Perceive}, \emph{Know}, \emph{Transform}
  \item \textbf{Nouns}: \emph{Air}, \emph{Earth}, \emph{Fire}, \emph{Water}, \emph{Animals}, \emph{Plants}, \emph{Body}, \emph{Illusion}, \emph{Mind}, \emph{Arcana}
\end{itemize}
The bonus for each \Specialization{} can be $0$, $+1$, or $+2$ (for starting characters) and may increase up to $+4$ when a character levels up.

Each \Specialization{} has has an associated tag that specify the limits of the character's ability with that \Specialization.
These tags are:\index{Magic|see{Specialization}}
\begin{itemize}
  \item \textbf{Human}\index{Specialization!Human}: the character can only perform within normal human ability.
  \item \textbf{Superhuman}\index{Specialization!Superhuman}: the character can perform within normal human ability easily ($-1$ \Difficulty\footnotemark) and can perform with superhuman ability (though within the laws of nature).
  \item \textbf{Simple \& Weak}\index{Specialization!Simple \& Weak}: the character can perform simple acts of magic easily ($-1$ \Difficulty) and can perform trivial acts of magic without a roll; the extent of the ability must be narrowly defined between the Player and GM.
  \item \textbf{Complex \& Powerful}\index{Specialization!Complex \& Powerful}: the character can perform complex acts of magic but any act of magic is difficult ($+1$ \Difficulty); the extent of the ability is widely defined and should be pushed to the limits. Failures on rolls to perform magic often result in magic gone awry.
\end{itemize}
\footnotetext{On an opposed roll, if a creature has a $-1$ \Difficulty{} they add a $+3$ bonus to their roll and if they have a $+1$ \Difficulty{} they add a $-3$ penalty to their roll.}
In general, a tag cannot be changed after character creation unless the \Specialization{} only had a bonus of $+0$ and the \emph{Human} tag in which case the new tag represents learning a new ability.
The GM may allow changing tags in other situations as makes sense with the story.

Note that with magic setting the base difficulty of the roll depends on how magic is being used.
If the use of magic is ancillary to the task, for example a character is conjuring fire in their hand in an attempt to intimidate a Lord, then the base difficulty is the difficulty of the task they are attempting to accomplish, in our example intimidating the Lord.
If, instead, the use of magic is essential to the task, especially if the task could not be accomplished without magic, then the base difficulty is generally \emph{Medium}.
In both cases, the GM may adjust the difficulty as appropriate to the situation; for example they may increase the difficulty if there is limited time or if the Lord is familiar with magic and unlikely to be spooked, or may decrease the difficulty if, for instance, the Lord has never seen magic before.

\subsubsection{Big Magic}\label{sec:big-magic}\index{Magic!Big Magic}
While the \emph{Complex \& Powerful} tag provides a character access to powerful and broad magic, there should still be things beyond their capacity.
When starting a campaign, a group should discuss what kinds of magics are beyond the capability of this tag; some examples that may be beyond the tag are magics that are permanent or have a long lasting impact and magic that has a large area of impact or impacts a large number of targets.
Magic beyond what be achieved using the \emph{Complex \& Powerful} tag is the realm of \textbf{Big Magic}.

To perform Big Magic requires more time, ability, and possibly resources than normal magic.
First, to undertake Big Magic, at least one character involved must have a relevant specialization with the \emph{Complex \& Powerful} tag because the ability to perform Big Magic still relies on some ability to perform magic.
Next, the GM may determine that a particular act of Big Magic requires some resources: these can be \emph{common} resources in which case the GM picks a wealth value required to acquire them or \emph{rare} resources which would require some kind of quest to acquire.
For \emph{rare} resources, instead of going on a new quest the GM may allow the use of some number of XP to cover the resource cost, but this would only be done when the Big Magic being worked on will have a permanent impact on the party or important ally.
Once any necessary resources have been acquired, performing Big Magic requires a Collective Roll, often with a high magnitude.
Generally the Difficulty of the roll is Hard (because it relies on the \emph{Complex \& Powerful} tag) but the GM may chose to increase or decrease this Difficulty.
Each roll as part of the Collective roll represents preparations and efforts to perform the magic and it can be assisted by any number of creatures, only one of whom needs to be rolling for a relevant \Specialization{}.
Generally each roll represents about two hours of work, though the GM may increase or decrease this time as appropriate, for example a group of experienced practitioners may be faster while novice ones may be slower.

On rolls for Big Magic, bringing additional materials and resources, especially ones connected to the magic being performed, can be used, at the GMs discretion, to gain an \Advantage{} for the roll.

\subsection{Experiences}
\emph{Experiences}\index{Experience} provide additional depth to a character by detailing parts of their past that may provide useful experience in circumstances they encounter in adventuring.
Each \Experience{} has a short, couple word, description and a bonus attached to; when a character is attempting something related to one of their \emph{Experiences} they may spend an \Advantage{} to activate it and add the \emph{Experience's} bonus to their roll.
Good experiences will be broad enough that there are a number of possible circumstances they could apply but they should not be able to apply in all circumstances.
Work with your GM and group to determine the appropriate breadth of \emph{Experiences} for your game and to decide when an \Experience{} applies.
The bonus associated with an \Experience{} is $+1$ or $+2$ for newly created characters and can be increased to $+3$ when a character levels up.

\subsection{Abilities}
\emph{Abilities}\index{Ability} provide additional unique mechanics for characters, generally to do with gaining or using \Advantage.
While not solely for non-magical characters, these \Ability{} are designed to assist non-magical builds by offering additional options for play and ways to gain \Advantage{} which allows the more use of \emph{Experiences}.
Below is a list of such abilities, along with their XP costs; players and GMs should feel free to build other \emph{Abilities} to match their desires.
Several of these abilities have a \emph{level} $n$ which, for example, impacts the number of \Advantage{} that a character gains; to take the ability you at first gain it at level 1 and can then spend the specified cost to increase it to level 2 and so on.
When acquiring an ability whose cost depends on $n$, you must spend XP based on the level ($n$) you are upgrading to, so \textbf{Take the Advantage} costs 2 XP for level 1, another 3 XP for level 2, and so on.
\begin{itemize}
  \item \textbf{Take the Advantage} ($n$ XP): At the beginning of each session, gain $n$ \Advantage.
  \item \textbf{Good Under Pressure} ($n$ XP): Once per session, ask the GM how many \Advantage{} they have; if they have more than all Player Characters combined, gain $n$ \Advantage.
  \item \textbf{Tank} ($n$ XP): When hit by attack, you can spend \Advantage{} to decrease the damage you take by $n$ per \Advantage{}.
  \item \textbf{Natural Healer} ($n$ XP): Each session, your first $n$ attempts to triage creatures (as described in Section~\ref{sec:triage}) do not require spending an \Advantage.
  \item \textbf{Powerful Healer} ($n$ XP): Each \Advantage you spend to do additional healing when you triage a creature heals an additional $n$ \Life.
  \item \textbf{Draw Blood} ($n$ XP): When you hit a target with an attack, you can spend two \Advantage{} to inflict a \emph{bleed} on the target. The target takes $n$ damage at the beginning of each of their turns until they spend an action or spend an \Advantage{} to end the condition.
\end{itemize}

\subsection{Combat Styles}\label{sec:combat-styles}
\emph{Combat Styles}\index{Combat Style} define the character's abilities in combat, including their bonus to attack and defend, their range on attacks, their damage die, and other special abilities.
Combat Styles are divided into \emph{Offensive} and \emph{Defensive} styles.
Except at character creation, gaining a new Combat Style costs 3 XP.
Each Combat Style, when chosen, is associated with a particular Basic Statistic which forms the base of the bonus for the style; when an additional Combat Style is chosen using the same statistic as another Combat Style the character already has, an additional 3 XP must be paid per such Combat Style (i.e., if a character has two Combat Styles that use Finesse then 6 XP must be paid to add a third Combat Style using Finesse).
For Combat Styles that provide a bonus to attack or defend, this bonus can be increased for the same XP costs as a \emph{Specialization} and is subject to the same maximum cap (as described below).
For Offensive Combat Styles, the number of damage dice it grants can be increased by 1 for a number of XP equal to $3 \times$ the new number of damage die (after the increase), so an increase from 1 die to 2 dice costs 6 XP.
Some Combat Styles have other upgrade options that are described with them.
The Combat Style options are detailed below:

\subsubsection{Offensive Styles}\index{Combat Style!Offensive}
\paragraph{Melee}
Choose a Basic Statistic as the base attack bonus; the character gains a $+1$ bonus to attacks with this style on top of the Basic Statistic's bonus.
The range of attacks with this style is \emph{in hand-to-hand combat}.
The damage die for attacks with this style is d8; and attacks with this style deal a base of 0 damage die.
Describe a new Experience (ideally related in some way to the character's combat abilities this style represents, for example related to knowledge of weapons, combat tactics, etc.); this Experience has a bonus equal to the attack bonus (not including the Basic Statistic) granted by this style.
\\
\underline{Upgrades:}
\begin{itemize}
  \item[-] The range can be changed to \emph{just oustide the reach of a similarly sized creature} for 2 XP.
\end{itemize}

\paragraph{Ranged}
Choose a Basic Statistic as the base attack bonus; the character gains a $+1$ bonus to attacks with this style on top of the Basic Statistic's bonus.
The range of attacks with this style is \emph{within a reasonable distance}.
The damage die for attacks with this style is d6; and attacks with this style deal a base of 0 damage die.
Describe a new Experience (ideally related in some way to the character's combat abilities this style represents, for example related to knowledge of weapons, combat tactics, etc.); this Experience has a bonus equal to the attack bonus (not including the Basic Statistic) granted by this style.
\\
\underline{Upgrades:}
\begin{itemize}
  \item[-] The range can be changed to \emph{from an unreasonable distance} for 2 XP.
  \item[-] The damage die for this style can be changed to d8 for 5 XP.
\end{itemize}

\paragraph{Simple \& Weak}
Choose a Basic Statistic as the base attack bonus; choose a \emph{Specialization} with the Simple \& Weak tag, this Specialization is the attack bonus with this style (on top of the Basic Statistic's bonus).
The range of attacks with this style is \emph{nearby}, meaning that the attacker can be targeted by both ranged and melee attacks from their target.
The damage die for attacks with this style is d4; and attacks with this style deal a base of 0 damage die.
\\
\underline{Upgrades:}
\begin{itemize}
  \item[-] The range can be changed to either \emph{in hand-to-hand combat} or \emph{within a reasonable distance} for 2 XP.
  \item[-] The damage die for this style can be changed to d6 for 5 XP.
\end{itemize}

\paragraph{Complex \& Powerful}
Choose a Basic Statistic as the base attack bonus; choose a \emph{Specialization} with the Complex \& Powerful tag, this Specialization is the attack bonus with this style (on top of the Basic Statistic's bonus).
The range of attacks with this style is \emph{from an unreasonable distance}, but for attacks from a long range an additional $-1$ penalty should be applied to the roll.
The damage die for attacks with this style is d10; and attacks with this style deal a base of 0 damage die.
\\
\underline{Upgrades:}
\begin{itemize}
  \item[-] The penalty for attacks at long range can be eliminated for 2 XP.
\end{itemize}

\subsubsection{Defensive Styles}\index{Combat Style!Defensive}
\paragraph{Evasive}
Choose a Basic Statistic as the base bonus to defend; the character gains a $+1$ bonus to defend with this style on top of the Basic Statistic's bonus.

\paragraph{Armored}
Choose a Basic Statistic as the base bonus to defend.
The character also gains a $+1$ bonus to Block (this bonus can be increased, up to a maximum of $+2$ at character creation or $+5$ thereafter, for XP equal to $3\times$ the new bonus).

\paragraph{Shielded}
Choose a Basic Statistic as the base bonus to defend.
When hit by an attack, the character can spend an Advantage to reduce the damage by this Basic Statistic.

\subsection{Character Creation and Leveling-Up}
When creating a new character for a campaign, the GM should specify the number of \emph{Experience Points} (XP) that the players have to start with to build their characters.
A starting character (before spending any XP) has a $+0$ bonus for each of their basic statistics, a \Life{} of 20, a \Recovery{} of 10, a \Block{} of 0, a \emph{Speed} of \emph{Normal} (and only the ability to walk), and a bonus of $+0$ and the \emph{human} tag for all \emph{Specializations}.
A starting character also has one Offensive and one Defensive Combat Styles (for these to have the same Basic Statistic still costs 3 XP). 
A starting character also has one \emph{Item} whose construction does not cost XP beyond the cost of the \emph{Item} itself (see Section~\ref{sec:items} for details).
These statistics can be increased by spending XP, and in some cases decreased to gain XP, as follows:
\begin{itemize}
  \item To increase a basic statistic (to a maximum of $+2$) by $1$ costs $3 \times$ the score after the bonus; so increasing from $+0$ to $+1$ costs 3 XP, and from $+1$ to $+2$ costs 6.
    A basic statistic can also be decreased to $-1$ to gain 3 XP.
  \item To increase the character's Total \Life{} costs 1 XP for each point up to 30 Total \Life{} and then costs 2 XP for each point up to the maximum of 40.
    The Total \Life{} can also be decreased to gain XP, gaining 1 XP per point down to 10 and then 2 XP per point down to a minimum of 1.
  \item To increase the character's \Recovery{} costs 2 XP for each point up to 15 and then 4 XP for each point up to the maximum of 20.
    \Recovery{} can also be reduced to gain XP, gaining 2 XP for each point down to 5 and 4 XP for each point down to the minimum of 1.
  \item Increasing the character's \Block{} (up to a maximum of 2) costs $3 \times$ the new block; so a block of 1 costs 3 XP and increasing that to 2 costs 6 XP.
  \item A character's \Speed{} can be increased to \emph{Fast} for 2 XP or can be reduced to \emph{Slow} to gain 2 XP.
  \item To increase a \Specialization{} (up to a maximum of $+2$) bonus by $+1$ costs the score after the bonus; so increasing from $+0$ to $+1$ costs 1 XP and from $+1$ to $+2$ costs 2.
  \item To add a tag to a \Specialization{} (that only has the default \emph{human} tag costs the following:
    \begin{itemize}
      \item \textbf{Superhuman}: 3 XP
      \item \textbf{Simply \& Weak}: 2 XP
      \item \textbf{Complex \& Powerful}: 2 XP
    \end{itemize}
  \item To gain a $+1$ \Experience{} costs 1 XP and then increasing it (to a maximum of $+2$) by $1$ costs the score after the increase; so increasing a $+1$ experience to a $+2$ experience costs 2 XP.
  \item To gain an \Ability{}.
  \item To upgrade a \emph{Combat Style} or gain a new one as described in Section~\ref{sec:combat-styles}.
  \item To add additional features to their initial item or to construct additional items as described in Section~\ref{sec:items}.
\end{itemize}

Throughout their journeys, characters will also earn XP that they can use to improve their characters.
These may be spent in the same ways as above, but can also be spent to:
\begin{itemize}
  \item Increase a basic statistic to $+3$; cost is as described above.
  \item Increase a character's Total \Life{} beyond 40; the cost of each increase is one additional XP for each 10 points, so each point up to 50 costs 3 XP, up to 60 costs 4 XP, and so on.
  \item Increase a character's \Recovery{} beyond 20; the cost of each increase is two additional XP for each 5 points, so each point up to 25 costs 6 XP, up to 30 costs 8 XP, and so on.
  \item Increase a character's \Block{} up to 5; cost is as described above.
  \item Increase a \Specialization{} bonus up to $+4$; cost is as described above.
  \item Increase an \Experience{} bonus up to $+3$; cost is as described above.
\end{itemize}
